\documentclass[12pt]{book}
\setlength{\headheight}{15pt}

\usepackage[margin=1in]{geometry}
\usepackage{fancyhdr}
\usepackage{amssymb}
\usepackage{lastpage}

% begin custom header shit
\pagestyle{fancy}
\fancyhf{}
\lhead{COMP2650 Lecture 3 - Number Systems II}
\rhead{Isaac Kilbourne}
\rfoot{Page \thepage\ of \pageref{LastPage}}
% end custom header shit

\newenvironment{indented}[1] {
	\begin{list}{}{\setlength{\leftmargin}{#1}}
		\item[]
}{\end{list}}

\begin{document}
	\noindent
	1. How many positions are needed to represent the following numbers in base-2 with no error?
	
	\begin{indented}{5mm}
		a. (1.5)$_{10}$
		\begin{indented}{5mm}
			$\log_2(0.5) = 1$ fraction bit

			\medskip
			$\therefore$ only 2 bits (one integer, one fraction) are needed to accurately 
			represent (1.5)$_{10}$: 
			
			\ \ \ (1.1)$_2 = (1 \times 2^0) + (1 \times 2^{-1}) = (1.5)_{10}$
		\end{indented}
	\end{indented}

	\begin{indented}{5mm}
		b. (1.05)$_{10}$
		\begin{indented}{5mm}
			Impossible to store with complete accuracy, since $\log_2(0.05) \approxeq -4.321$ 
			is not an integer. This indicates there is no finite number of bits that can be picked
			to store it. Someone call the IEEE wizards!
		\end{indented}
	\end{indented}

	\noindent
	2. Given n=4 positions, put the fraction point in a position for the following numbers in 
	base-2 to minimize the conversion error.

	\begin{indented}{5mm}
		a. (1.5)$_{10}$
		\begin{indented}{5mm}
			xxx.x will have precision 0.5 (2$^{-1}$), thus at least one bit must be allocated
			to the fraction to have perfect precision. Thus, the following are all valid
			and completely accurate representations:
			
		\begin{list}{}{}
			\item- (001.1)$_2$
			\item- (01.10)$_2$
			\item- (1.100)$_2$
		\end{list}

		\end{indented}
	\end{indented}

	\begin{indented}{5mm}
		b. (5.5)$_{10}$
		\begin{indented}{5mm}
			xxx.x will still have precision 0.5 (2$^{-1}$), so at least one fraction bit
			is required, however (5)$_{10}$ requires 3 bits so the only perfect representation
			is (101.1)$_2$.
		\end{indented}
	\end{indented}

	\begin{indented}{5mm}
		c. (10.5)$_{10}$
		\begin{indented}{5mm}
			The glaring issue with this number is that there are only 4 bits to represent the
			number, but 4 bits are needed for the integer. Thus, the only solution is to minimize
			error. 0.5 is less than 3 (10 - 7, the max for a 3-digit binary
			number), so we must forego the fraction bit. Therefore the most accurate binary
			representation possible is (1010.)$_2$.
		\end{indented}
	\end{indented}
\end{document}