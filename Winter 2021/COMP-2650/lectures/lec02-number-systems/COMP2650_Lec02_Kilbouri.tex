\documentclass[12pt]{book}
\setlength{\headheight}{15pt}

\usepackage[margin=1in]{geometry}
\usepackage{fancyhdr}
\usepackage{amssymb}
\usepackage{lastpage}

% begin custom header shit
\pagestyle{fancy}
\fancyhf{}
\lhead{COMP2650 Lecture 2 - Number Systems I}
\rhead{Isaac Kilbourne}
\rfoot{Page \thepage\ of \pageref{LastPage}}
% end custom header shit

\newenvironment{indented}[1] {
	\begin{list}{}{\setlength{\leftmargin}{#1}}
		\item[]
}{\end{list}}

\begin{document}
	\noindent
	4. Convert the following numbers to binary:


	\begin{indented}{5mm}
		a. (5F.A2)$_{16}$
		\begin{indented}{5mm}
			$\begin{array}{c|c|c|c|c}
				(5)_{16} \to 5 & ($F$)_{16} \to 15 & . & ($A$)_{16} \to 10 & (2)_{16} \to 2\\
				0101 & 1111 & . & 1010 & 0010
			\end{array}$

			$\therefore$ (5F.A2)$_{16}$ = (0101 1111 . 1010 0010)$_2$
		\end{indented}
	\end{indented}

	\begin{indented}{5mm}
		b. (213.32)$_4$
		\begin{indented}{5mm}
			$\begin{array}{c|c|c|c|c|c}
				(2)_4 \to 2 & (1)_4 \to 1 & (3)_4 \to 3 & . & (3)_4 \to 3 & (2)_4 \to 2\\
				10 & 01 & 11 & . & 11 & 10
			\end{array}$

			$\therefore$ (213.32)$_4$ = (10 01 11 . 11 10)$_2$
		\end{indented}
	\end{indented}

	\noindent
	5. Obtain the 1's and 2's complements of the following binary numbers by showing the steps:
	\begin{indented}{5mm}
		a. 10010000
		\begin{indented}{5mm}
			1s complement: 01101111 (found by simply inverting the bits).
			
			2s complement: 01110000 (found by adding one to the 1s complement)
		\end{indented}
	\end{indented}

	\begin{indented}{5mm}
		b. 00000000
		\begin{indented}{5mm}
			1s complement: 11111111 (inverting all bits)
			
			2s complement: (1)00000000 (adding one to the 1s complement)
		\end{indented}
	\end{indented}

	\begin{indented}{5mm}
		c. 11111111
		\begin{indented}{5mm}
			1s complement: 00000000 (inverting all bits)

			2s complement: 00000001 (adding one to the 1s complement)
		\end{indented}
	\end{indented}
\end{document}